\addcontentsline{toc}{chapter}{Abstract}

\begin{center}
\Large  {\bf Abstract }
\end{center}
Larval crowding has been shown to influence evolution of higher competitive ability in \textit{Drosophila}. However the effect of ecological dynamics in larval crowding environment on the evolution of larval traits is relatively less explored.\\\\
Here I describe a stage-structured individual-based model to investigate the evolution of larval traits in \textit{Drosophila melanogaster} populations adapted for various larval crowding conditions. The model also describes the ecological dynamics during larval feeding by simulating feeding band dynamics and the diffusion of metabolic waste from the feeding band into the food below. The model is parameterized using empirical data on multiple Drosophila laboratory populations adapted for larval crowding. The model simulates the effect of the amount of food and number of larvae on the evolution of greater competitive ability. It is also used to observe interactions among larval traits such as larval feeding rate, efficiency to convert food into biomass, critical size, waste tolerance, time to reach critical size and body size.\\\\
A further simulation study is focused on the effect of heritability and initial standing variation (sources of stochasticity in the model) on the evolutionary trajectories of larval traits responsible for competitive ability. I further extend the model to investigate patterns of early-late eclosing larvae and the larval traits they exhibit across densities. Results from this simulation study give a better understanding of various factors involved in the adaptation of \textit{D. melanogaster} populations subjected to various scenarios of larval crowding.
\vfill
\cleardoublepage
