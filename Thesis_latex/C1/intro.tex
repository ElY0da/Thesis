\chapter{Introduction}
The theory of density-dependent natural selection was verbally introduced by \citet{macarthurGENERALIZEDTHEOREMSNATURAL1962}, \citet*{macarthurTheoryIslandBiogeography1967} to explore the evolution of phenotypes dependent on the population density. It is considered to be a critical link between ecological and evolutionary dynamics \citep{muellerTheoreticalEmpiricalExamination1997}. Over many years, this theory has been modified mathematically and studied experimentally to understand density effects on the evolution in great detail \citep{andersonDensityRegulatedSelectionGenotypic1983,asmussenDensityDependentSelectionIncorporating1983,clarkeDensityDependentSelection1972,muellerTheoreticalEmpiricalExamination1997,roughgardenDensityDependentNaturalSelection1971,santosDensityDependentNaturalSelection1997}. Early experimental studies have also shown that selection at extreme densities causes selection for higher competitive ability since there is competition for limited resources \citep{joshiKselectionAselectionEffectiveness2001}. Such competitive ability of an organism is composite phenotype determined by various life-history traits. Through these early studies, it was clear that density is a significant factor in determining the life-history of organisms which are essential for competitive ability. Competition plays a vital role in determining not only evolutionary outcomes of species but also ecological outcomes which affect population dynamics and interactions with other species \citep{caseIllustratedGuideTheoretical2000a,deyAdaptationLarvalCrowding2012}. Thus, in order to grasp a better understanding of density-dependent selection, exploring the effect of competitive ability on ecological and evolutionary dynamics becomes essential \citep{prasadWhatHaveTwo2003}.  \\\\
Over the last four decades, various \textit{Drosophila melanogaster} laboratory populations have been used to study the evolution of life-history traits due to density-dependent selection. One of the first experimental evolution studies used r and K populations of \textit{D. melanogaster} \citep{muellerTradeoffRselectionKselection1981} to investigate the \textit{r}- and \textit{K}-selection theory by \citet{macarthurGENERALIZEDTHEOREMSNATURAL1962}, \citet*{macarthurTheoryIslandBiogeography1967}. In these populations, \textit{r}-selection lines were maintained at low density giving density-independent selection, whereas lines for \textit{K}-selection were maintained at extreme densities such that selection was density-dependent. As predicted by early mathematical models, these studies showed that \textit{r}-selected populations favoured traits responsible for higher population growth rate at low densities but lower growth rate at extreme density. \citet{bakkerAnalysisFactorsWhich1962,burnetGeneticAnalysisLarval1977} suggested that larval feeding rate, which is measured as retraction rate of cephalopharyngeal sclerites of the larva, is a key factor in larval competitive ability. Experimental studies on \textit{r}- and \textit{K}-selection showed that \textit{K}-selected populations have higher competitive ability along with increased larval feeding rate \citep{joshiEvolutionHigherFeeding1988}. This lead to the conclusion that larval feeding rate is a good measure of competitive ability in \textit{Drosophila} larvae. These populations could not predict classical density-dependent outcomes such as higher efficiency of food into biomass conversion \citep{muellerDensitydependentNaturalSelection1990}. Another problem with these populations was that \textit{r} populations were maintained in discrete generation cycles while \textit{K} populations were maintained in overlapping generations. \\\\
The successive experimental evolution studies were aimed at tackling questions raised in experimental studies mentioned above, by having stage-specific density-dependent selection in a new set of \textit{D. melanogaster} populations (described in \cite{joshiDirectionalStabilizingDensityDependent1993}). In this long-term evolution study, a set of larval crowding (CU) population, another set of adult crowding (UC) population and one set of uncrowded (UU) \textit{D. melanogaster} population were used. CU population adaptated to larval crowding through similar set of traits seen in \textit{K} populations. CU population had higher competitive ability than UU population at high density but did not show trade-off similar to \textit{K} population across densities \citep{borashPatternsSelectionStress2001,joshiDirectionalStabilizingDensityDependent1993,santosDensityDependentNaturalSelection1997}. Such competitive ability in CU larvae was due to increased feeding rate and increased nitrogenous waste tolerance at the cost of poor efficiency to convert food into biomass \citep{borashGeneticPolymorphismMaintained1998,joshiDensitydependentNaturalSelection1996,shiotsuguSymmetryCorrelatedSelection1997}. These results established a canonical view of density-dependent selection in \textit{Drosophila} which argued that the evolution of greater competitive ability occured through increased feeding rate and metabolic waste tolerance at the cost of efficiency of food utilization \citep{joshiKselectionAselectionEffectiveness2001}. \\\\
After the canonical view on adaptation to larval crowding was accepted widely, recent studies in different \textit{Drosophila} species questioned this view. A subsequent study on adaptation to larval crowding involved \textit{D. ananassae} and \textit{D. nasuta nasuta} species of \textit{Drosophila} which were subjected to long-term selecion experiments similar to CU-UU populations \citep{nagarajanAdaptationLarvalCrowding2016}. These populations differed with CU population such that these were maintained at similar larval density but with decreased absolute number of eggs and total larval food.     
\pagebreak
