\chapter{Introduction}
The theory of density-dependent natural selection was verbally introduced by \textcite{macarthurTheoryIslandBiogeography1967} to explore the evolution of phenotypes dependent on the population density. Over many years, this theory has been modified mathematically and studied experimentally to understand density effects on the evolution (ref). Density-dependent selection is an important link which connects population ecology and population genetics which had largely remained separate \parencite{muellerTheoreticalEmpiricalExamination1997}. Early experimental studies have also shown that selection at high densities causes selection for higher competitive ability \parencite{joshiKselectionAselectionEffectiveness2001}. Competition plays major role in not only evolutionary outcomes of a species but also ecological outcomes affecting interactions other species and population dynamics (\cite{caseIllustratedGuideTheoretical2000a}; \cite{deyAdaptationLarvalCrowding2012}). Thus, in order to grasp a better understanding of density-dependent selection, exploring the effect of competitive ability on ecological and evolutionary dynamics becomes an important fector.   
\pagebreak
