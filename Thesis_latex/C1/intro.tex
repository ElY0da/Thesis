\chapter{Introduction}
The theory of density-dependent natural selection was verbally introduced by \citet{macarthurGENERALIZEDTHEOREMSNATURAL1962}, \citet*{macarthurTheoryIslandBiogeography1967} to explore the evolution of phenotypes dependent on the population density. It is considered to be a critical link between ecological and evolutionary dynamics \citep{muellerTheoreticalEmpiricalExamination1997}. Over many years, this theory has been modified mathematically and studied experimentally to understand density effects on the evolution in great detail \citep{andersonDensityRegulatedSelectionGenotypic1983,asmussenDensityDependentSelectionIncorporating1983,clarkeDensityDependentSelection1972,muellerTheoreticalEmpiricalExamination1997,roughgardenDensityDependentNaturalSelection1971,santosDensityDependentNaturalSelection1997}. Early experimental studies have also shown that selection at extreme densities causes selection for higher competitive ability since there is competition for limited resources \citep{joshiKselectionAselectionEffectiveness2001}. Such competitive ability of an organism is composite phenotype determined by various life-history traits. Through these early studies, it was clear that density is a significant factor in determining the life-history of organisms which are essential for competitive ability. Competition plays a vital role in determining not only evolutionary outcomes of species but also ecological outcomes which affect population dynamics and interactions with other species \citep{caseIllustratedGuideTheoretical2000a,deyAdaptationLarvalCrowding2012}. Thus, in order to grasp a better understanding of density-dependent selection, exploring the effect of competitive ability on ecological and evolutionary dynamics becomes essential \citep{prasadWhatHaveTwo2003}.  \\\\
Over the last four decades, various \textit{Drosophila melanogaster} laboratory populations have been used to study the evolution of life-history traits due to density-dependent selection. One of the first experimental evolution studies used r and K populations of \textit{Drosophila melanogaster} \citep{muellerTradeoffRselectionKselection1981} ) to investigate the \textit{r}- and \textit{K}-selection theory by  \citet{macarthurGENERALIZEDTHEOREMSNATURAL1962}, \citet*{macarthurTheoryIslandBiogeography1967}. In these populations, \textit{r}-selection lines were maintained at low-density, giving density-independent selection. In contrast, lines for \textit{K}-selection were maintained at extreme densities such that selection was density-dependent. As predicted by early mathematical models, these studies showed that \textit{r}-selected populations favoured traits responsible for higher population growth rate at low densities but lower growth rate at extreme density. \citet{bakkerAnalysisFactorsWhich1962,burnetGeneticAnalysisLarval1977} suggested that larval feeding rate, which is measured as retraction rate of cephalopharyngeal sclerites of the larva, is a critical factor in larval competitive ability. Experimental studies on \textit{r}- and \textit{K}-selection showed that \textit{K}-selected populations have higher competitive ability along with increased larval feeding rate \citep{joshiEvolutionHigherFeeding1988}. This lead to the conclusion that larval feeding rate is a good measure of competitive ability in \textit{Drosophila} larvae. These populations could not predict classical density-dependent outcomes such as higher efficiency of food into biomass conversion \citep{muellerDensitydependentNaturalSelection1990}. Another problem with these populations was that r populations were maintained in discrete generation cycles while K populations were maintained in overlapping generations. \\\\
The successive experimental evolution studies were aimed at tackling questions raised in experimental studies mentioned above, by having a stage-specific density-dependent selection in a new set of \textit{D. melanogaster} populations (described in \cite{joshiDirectionalStabilizingDensityDependent1993}). In this long-term evolution study, a set of larval crowding (CU) population, another set of adult crowding (UC) population and one set of uncrowded (UU) \textit{D. melanogaster} population were used. CU population adapted to larval crowding through a similar set of traits seen in K populations. CU population had higher competitive ability than UU population at high-density which leads to increased pre-adult survivorship and decreased pre-adult development time \citep{borashPatternsSelectionStress2001,joshiDirectionalStabilizingDensityDependent1993,santosDensityDependentNaturalSelection1997}. Such competitive ability in CU larvae was due to increased feeding rate and increased nitrogenous waste tolerance at the cost of poor efficiency to convert food into biomass \citep{borashGeneticPolymorphismMaintained1998,joshiDensitydependentNaturalSelection1996,shiotsuguSymmetryCorrelatedSelection1997}. These results established a canonical view of density-dependent selection in \textit{Drosophila} which argued that the evolution of greater competitive ability occured through increased feeding rate and metabolic waste tolerance at the cost of efficiency of food utilization \citep{joshiKselectionAselectionEffectiveness2001}. \\\\
After the canonical view on adaptation to larval crowding was accepted widely, recent studies in different \textit{Drosophila} species questioned this view. A subsequent study on adaptation to larval crowding involved \textit{D. ananassae} and \textit{D. nasuta nasuta} species of \textit{Drosophila} which were wild-caught and subjected to long-term selection experiments similar to UU-CU populations \citep{nagarajanAdaptationLarvalCrowding2016}. UUnlike CU population these were maintained at similar larval density but with decreased absolute number of eggs and total larval food. Due to adaptation to larval crowding in these populations, there was an increase in pre-adult survivorship at high-density and faster development compared to control at both low and high-density. In contrast to results from previous K and CU populations, these populations showed a reduction in time to reach critical size with no increase in larval feeding rate nor in nitrogenous waste tolerance \citep{nagarajanAdaptationLarvalCrowding2016}. Such reduced minimum critical feeding time was speculated to be due to increased efficiency of food into biomass conversion, which fit the \textit{K}-selection theory of \citet*{macarthurTheoryIslandBiogeography1967}. These surprising results were thought to be an outcome of several factors such as differences in species-specific genetic architect of traits responsible for larval competitive ability, differences in wild-caught populations and long-term laboratory populations, as well as differences in maintains of larval crowding suggesting the effect of ecological factors.\\\\
A  long-term follow-up study on adaptation to larval crowding was performed using \textit{Drosophila melanogaster} populations derived from UU populations to answer the questions raised from larval crowding studies of \citet{nagarajanAdaptationLarvalCrowding2016}. In this study, a set of control  populations (MB: Melanogaster Baseline) which had low larval density, and another set of populations (MCU: Melanogaster Crowded as larvae Uncrowded as adults) where larval stage was maintained at high-density similar to larval crowded populations of \textit{D. ananassae} and \textit{D. nasuta nasuta} \citep{sarangiEvolutionIncreasedLarval2016}. MCU population showed the evolution of greater larval competitive ability through a similar set of traits observed in the study of \citet{nagarajanAdaptationLarvalCrowding2016}, i.e. decrease in the time to reach critical size without an increase in feeding rate. MCU larvae also did not differ in terms of metabolic waste tolerance but still showed faster pre-adult development time at both densities \citep{sarangiEvolutionIncreasedLarval2016}. In addition to these results, both MB and MCU populations showed a significant lower survivorship in the larval density of 1200 eggs / 6 ml food (CU-type culture) than in larval density of 600 eggs / 1.5 ml food (MCU-type culture) \citep{sarangiPreliminaryInvestigationsCauses2013}. MCU and CU population were derived from the same ancestry but still showed differences, indicating that ecological factors such as the overall number of eggs and total larval food, would be playing a significant role in determining which traits are selected for achieving greater competitive ability under larval crowding. \\\\
A subsequent study exploring ecological factors affecting adaptation to larval crowding involved two new set \textit{D. melanogaster} populations derived from MB population. One set of these populations was CCU (Control Crowded as larvae Uncrowded as adults) population to address the effect of the absolute number of eggs and total food on evolution od larval competitive ability. Another set of populations was LCU (Larry Mueller Crowded as larvae Uncrowded as adults) population aimed at controlling for the food differences between CU and MCU populations since larval food used in these populations were banana and cornmeal medium respectively. In all these four \textit{D. melanogaster} populations (MB, MCU, CCU and LCU) the adult stage was maintained in pretty much similar manner, whereas the details of larval stage maintenance are given in table~\ref{tab:larval_pop} \citep{sarangiEcologicalDetailsMediate2018}.
\begin{table}[h]
  \centering
  \begin{tabular}{|c|c|c|c|c|}
    \hline
    \textbf{No.} & \textbf{Population} & \textbf{No. of eggs} & \textbf{Food volume} & \textbf{Vial dimensions} \\
    \hline
    1. & MB & 70 & 6 ml & 9.5 cm \textit{h} $\times$ 2.4 cm \textit{d} \\
    \hline
    2. & MCU & 600 & 1.5 ml & 9.5 cm \textit{h} $\times$ 2.4 cm \textit{d} \\
    \hline
    3. & CCU & 1200 & 3 ml & 9.5 cm \textit{h} $\times$ 2.4 cm \textit{d} \\
    \hline
    4. & LCU & 1200 & 6 ml & 9.5 cm \textit{h} $\times$ 2.2 cm \textit{d} \\
    \hline
  \end{tabular}
  \caption{Larval stage maintenance in MB, MCU, CCU and LCU populations}
  \label{tab:larval_pop}
\end{table}\\
After several generations of selection, CCU and LCU populations showed an increase in competitive ability and had higher pre-adult survivorship compared to MB population at high densities \citep{sarangiEcologicalDetailsMediate2018}. These two populations showed much higher feeding rate along with no difference in nitrogenous waste tolerance for achieving greater competitive ability, unlike MCU population. These results were interesting since MCU and CCU populations were maintained at the same larval density with varying total number of eggs and food. \citet{sarangiEcologicalDetailsMediate2018} also showed that feeding rate measured at the third instar stage of these larvae was dependent on the number of larvae present during larval feeding when assayed in slial (slide vials) treatment. When assayed inside culture vials, it was observed that the overall feeding rate of MCU population was the highest in all three high-density treatments in contrast to previous results performed on petri-dish. This result suggested that the ecological dynamics of the culture vial does play an essential role in determining competitive ability. Inside a culture vial, larval feeding occurs only at the topmost part of total food present due to their inability to dig more \citep{godoy-herreraInterIntrapopulationalVariation1977}. This available upper part of the total food is approx 1 cm in the height of a standard vial used in MB, MCU and CCU populations, and is
called as 'feeding band' \citep{sarangiEcologicalDetailsMediate2018}. Thus, the sufficient larval density, i.e. number of larvae per feeding band is double in CCU population than in MCU population. Another significant finding regarding ecological dynamics inside a culture vial was the diffusion of metabolic waste excreted by larvae from the feeding band into the food below \citep{sarangiEcologicalDetailsMediate2018}. In MCU culture vials, the total amount of food is almost similar to the size of the feeding band. This lead to the speculation that in MCU culture vial, there is little-to-no diffusion of metabolic waste from the feeding band and food quality may decrease very rapidly during larval feeding affecting competitive ability. In CCU and LCU culture vials, diffusion of such metabolic waste occurs from the feeding band which leads could lead to a slower decrease in food quality affecting competitive ability in a manner different than in MCU culture vial.\\\\
In MCU, CCU and LCU populations, apart from pre-adult survivorship and feeding rate, other life-history traits such as dry weight at eclosion, development time also evolved differently \citep{sarangiEcologicalDetailsMediate2018}. Experimental studies are limiting in order to understand above-mentioned ecological factors inside culture vials of these populations. Thus, a computational simulation approach can be helpful to delve into ecological and evolutionary dynamics in these \textit{Drosophila} populations. \\\\
In this thesis, I have presented a precursory stage-structured individual-based model to investigate adaptation to larval crowding in different crowding conditions based on the study of \citet{sarangiEcologicalDetailsMediate2018}. This model is aimed at linking various ecological factors inside a culture vial, with the evolution of fitness-related traits and greater competitive ability through a combination of various larval traits. The later part of the model is also used to explore the role of initial standing variation in the population as well as heritability of larval traits, in determining the evolutionary trajectories to achieve greater competitive ability. \\\\
\pagebreak
