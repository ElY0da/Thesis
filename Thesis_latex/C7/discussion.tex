\chapter{Discussion}
Simulation results from the stage-structured individual-based model, which describes the ecological and evolutionary dynamics of \textit{Drosophila melanogaster} populations adapted for various larval crowding conditions, are presented in this thesis.\\\\
In Larval stage simulations, it is evident that waste build-up in the feeding band is dependent on effective larval density (number of eggs per feeding band) rather total larval density (chapter 2). Both MCU and CCU cultures have the same larval density but still, show different patterns of ecological dynamics inside the vial. Such a pattern might be one of the reasons, CCU populations evolved greater competitive ability through a set of larval traits different than observed in MCU populations. The model describes feeding band dynamics in the detail which is limited by experimental setups. In these larval stage simulations, CCU and LCU cultures show similar feeding band dynamics with waste build-up and food quality decrease till most of the larvae reach critical size, even though their overall larval densities are different. This result suggests that effective larval density plays a vital role in determining fitness-related traits such as larval body size, development time and survivorship, as suggested by \citep{sarangiEcologicalDetailsMediate2018}.\\\\
Further simulations performed on the larval stage showed the effect of larval trait parameters used in the model on fitness-related larval traits (chapter 3). These results describe the density-dependent effect of initial feeding rate, efficiency and critical size on the larval body size, development time and survivorship. These larval trait parameters interact differently in each culture due to differences in ecological dynamics. Such simulations help us to understand how a specific fitness-related trait value can be achieved through multiple sets of larval trait parameters \citep{sarangiEvolutionIncreasedLarval2016}.\\\\
One of the exciting results from the larval stage simulations shows that feeding rate at critical size is not only dependent on initial feeding rate but also on efficiency, critical size and effective larval density (chapter 3). The feeding rate at critical size in the model being a linear function of time to reach critical size, is dependent on several trait parameters as well as ecological dynamics. Such effect of interactions between larval trait parameters and larval density on the final feeding rate also explains the density-dependent behaviour of feeding rate seen in recent experimental studies. This model also explains how the recent empirical results showed MCU larvae with the highest feeding rate (post-critical) in culture vials \citep{sarangiEcologicalDetailsMediate2018}.\\\\
Later chapters in the thesis are focused on the evolutionary side of the model (chapter 4 and 5). In these simulation results, larval trait parameters in the model seem to evolve aggressively in CCU culture than in MCU culture simultaneously. From these evolved trait parameters, the fitness-related traits are obtained at each density and are similar to experimental MB and MCU populations but not CCU populations. This possibly suggests that there might be other factors playing a role at the evolutionary stage other than just the ecological differences between MCU and CCU. \\\\
I have used the model to explain the different evolutionary routes taken by these parameters are dependent on the variation which exists in the larval traits. In the model, the variation of each trait comes from initial standing variation and heritability. Simulations results from chapter 5 suggest that initial standing variation in critical size can affect the evolutionary trajectory of efficiency such that the efficiency of MCU population is different from CCU population. Initial standing variation also seems to positively correlate with maximum mean trait value a trait can attain after several generations of selection. Heritability in the model also seems to play a critical role in determining which trait evolves for greater competitive ability. These results show that heritability of one trait can affect the mean trait value of another trait in several cases of MCU and CCU populations differently. For example, the evolution of lesser critical size occurs only when heritability of efficiency is high along with lesser heritability of critical size in both MCU and CCU populations. In the model, MCU and CCU populations can have different trait values after several generations of selection if heritability and initial standing variation are calibrated enough. Results from these simulations suggest that ecological dynamics is just one part of the story where MCU and CCU populations differ and that these sources of variation in trait values might be playing a much more significant role than previously thought. Further experimental studies regarding heritability of larval traits are required to understand the differences competitive ability of MCU and CCU populations.\\\\
The last part where the model is focused on is the investigation of the larval trait distribution with development time. The simulations for these trait distributions are aimed at exploring the patterns of trait polymorphism present in early-late eclosing larvae (chapter 6). The results show that early eclosing larvae have a lesser final feeding rate and critical size at different densities. Late eclosing larvae seem to show higher waste tolerant, and such polymorphism is maintained in all four populations without displaying overall change in mean waste tolerance. First set of simulations involved larval trait parameters which were independent of each other. The simulations at LCU density give U-shaped distribution of body size observed in experimental populations \citep{sarangiEcologicalDetailsMediate2018}. From the patterns of initial feeding rate, it was inferred that lower initial feeding rate of late eclosing larvae is responsible for their higher body size. Thus, the second set of simulations are performed with a negative correlation in mind to explain the presence of U-shaped body size distribution in MCU and CCU density as well. These preliminary simulation results suggest that all larval trait parameters exhibit polymorphism seen in early-late eclosing larvae, some of them might be correlated with each other, and they are responsible for determining U-shaped body size distribution in CCU and LCU populations across MCU, CCU and LCU densities.\\\\
The stage-structured individual-based model presented here, acts as a computational tool used for a better understanding of ecological and evolutionary dynamics of MB, MCU, CCU and LCU populations. The model simulates results which are mostly in correlation with empirical data. Since the model explores, various factors such as ecological details, initial standing variation, heritability, and polymorphism play a role in density-dependent selection studies so far. These factors are responsible for determining evolutionary routes taken to achieve greater competitive ability. The model is very flexible as several variables have been taken into account, and these variables can be refined or modified for the different crowded population. The sexually dimorphic traits considered in the models can also be studied in future to investigate how sex might play a role in different larval crowded conditions. The model is based on basic inheritance rules which can also be modified further. It is hoped that outcomes from this work will enable a better understanding of adaptation to crowding environment in not only \textit{Drosophila} but other holometabolous insects as well.
\pagebreak
