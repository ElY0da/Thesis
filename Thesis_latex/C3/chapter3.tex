\chapter{Interplay between Larval Trait Parameters and Life-history Traits}
In the base model of larval stage, trait parameters used are initial feeding rate, efficiency, critical size and waste tolerance. These parameters can not be measured directly via experimental approaches, but their effect on other larval traits such as body size, feeding rate at the third instar, development time can be measured easily. Here, I explore how larval trait parameters interact with each other and affect body size, time to reach critical size, feeding rate at critical size and survivorship. Since feeding rate in the model stays constant after reaching critical size, it can be taken as proxy for feeding rate at the third instar stage.

\section{Initial Feeding Rate and Efficiency}
In MB culture (low density), fig ~\ref{fig:fr vs eff mb} shows having higher efficiency as well as higher initial feeding rate gives higher larval body size, but lower time to reach critical size. Feeding rate at critical size is dependent on time taken to reach critical size which is dependent on body size increment at each time step. This body size increment is proportional to the current feeding rate and efficiency. Thus, efficiency and initial feeding rate both affect the feeding rate shown at the critical size. Having lower efficiency and higher initial feeding rate tends to give higher feeding rate at critical size in MB culture. Survivorship does not show any pattern at low density, since most of the larvae are competing very less and are able to survive easily. \\ \\
\begin{figure}[h]
  \centering
  \includegraphics[width=0.75\textwidth]{C3/Figs/Feeding rate_vs_Efficiency_MB}
  \caption{Effect of initial feeding rate and efficiency on larval traits in MB culture}
  \label{fig:fr vs eff mb}
\end{figure}\\
In MCU and CCU cultures (high densities), fig ~\ref{fig:fr vs eff mcu} and fig ~\ref{fig:fr vs eff ccu} show that time to reach critical size show similar pattern as seen in MB culture with varying efficiency and initial feeding rate. The maxima possible is higher in high density cultures than maxima possible in low density culture, showing that it takes more time to reach given critical size at high densities than at low density with same efficiency and initial feeding rate values. Feeding rate shown at critical size also shows similar pattern as seen in MB culture but with higher maxima reached with same parameter ranges. This suggests feeding rate shown at critical size is also a density dependent trait.
The complete white pixels in all heatmaps (fig ~\ref{fig:fr vs eff mcu} and fig ~\ref{fig:fr vs eff ccu}) are the values where none of the larvae survived, so the trait could not be measured and are to be excluded.
\\ \\
At high densities, body size and survivorship are seemed to be not affected by initial feeding rate, unlike at low density. Food acquired while having either higher or lesser initial feeding rate, remains almost the same. This is due to the decrease in food quality is higher for higher initial feeding rate. Thus, overall body size increment which is majorly determined by food quality at high densities, is approx. same in both cases i.e. with higher and lower initial feeding rate. Since survivorship is determined by whether critical size reached or not, it also shows similar pattern as body size for these two parameters.
\newpage
\begin{figure}
  \centering
  \includegraphics[width=0.75\textwidth]{C3/Figs/Feeding rate_vs_Efficiency_MCU}
  \caption{Effect of initial feeding rate and efficiency on larval traits in MCU culture}
  \label{fig:fr vs eff mcu}
\end{figure}
\begin{figure}
  \centering
  \includegraphics[width=0.75\textwidth]{C3/Figs/Feeding rate_vs_Efficiency_CCU}
  \caption{Effect of initial feeding rate and efficiency on larval traits in CCU culture}
  \label{fig:fr vs eff ccu}
\end{figure}\\


\pagebreak
%\renewcommand\bibname{{References}}
\bibliography{References}
\bibliographystyle{plain}
